\documentclass[11pt,a4paper,titlepage, oneside]{article}
\usepackage[utf8]{inputenc}
\usepackage{graphicx} 	% images
\usepackage{listings}
\usepackage{url}
\usepackage{times}
\usepackage[T1]{fontenc}      % caracteres francais
\usepackage{xcolor}		%couleurs
\usepackage[frenchb]{babel}  % langue
\usepackage{geometry}         % marges
\usepackage{verbatim}  
\usepackage{moreverb}       % texte preformate
\usepackage{array}
%\usepackage[linktocpage,colorlinks=true]{hyperref}
\usepackage[final]{pdfpages}

%Définition de l'apparence des listings
\lstset{
	frame=single,
	breaklines=true,
	basicstyle=\small,
	mathescape=false
}


%Le titre de projet
\title{Projet 2A STI : Supervision et audit de la sécurité système dans un réseau}

%Votre nom
\author{Aymeric Berquin \and Fayçal-Anoar Cherkaoui}

%Par défaut, Latex utilise la date du jour
%Pour supprimer la date ou la changer:
\date{1 Janvier 2015}
\geometry{top=3cm, bottom=3.5cm, left=3cm , right=2.5cm}


\begin{document}
%Créer la page de titre
\begin{titlepage}
 \thispagestyle{empty}
\begin{figure}[h]
  \centering
  \includegraphics[scale=0.75]{images/logo.png}
\end{figure}
\vspace{0,5cm}
\begin{center} 
\Huge{\textbf{{\color{red}}}Projet 2A STI : Supervision et audit de la sécurité système dans un réseau}
\\
\vspace{1.5 cm}

\vspace{1cm}
\large{Diplôme d'Ingénieur, 4e année}
\\
\vspace{1cm}
\large{Aymeric Berquin \\ \& \\ Fayçal-Anoar Cherkaoui}
\vspace{1 cm}
\paragraph{}
	\Large{Date de rendu de rapport : 01/01/2015}
	\\
	
\vspace{1.5 cm}
\end{center} 
\end{titlepage} 
\normalsize
\newpage
\section*{\textbf{Remerciement}}
\thispagestyle{empty}
	\paragraph{}
	Nous remercions Monsieur Briffaut pour le temps et les ressources qu'il nous a consacré. Nous le remercions aussi pour toutes les connaissances qu'il nous a apportés.

\newpage
\section*{{\color{red}Introduction}}
	\paragraph{toto}
		presentation du projet
		
		
\newpage
\thispagestyle{empty}
\tableofcontents
\listoffigures  % table des figures

\newpage
\pagenumbering{arabic} \setcounter{page}{1}
\section{{\color{red} Machines Virtuelles}}
\subsection{{\color{blue}{Installation du serveur}}}
	\paragraph{}
		Installation d'un debian classique.
		
\subsection{{\color{blue}{Installation du client}}}
	\paragraph{}
		Nous avons choisi d'utiliser un Xubuntu 14.04.
		Nom complet : user
		Nom d'utilisateur : user
		Mot de passe : resu
		Hostname : client1
		Nous n'avons rencontré aucun problème particulier.
\newpage
\section{{\color{red} Git}}
	\paragraph{}
		Nous avons décidé d'utiliser un serveur git pour gérer les sources.
	\paragraph{}
		
\newpage
\section{{\color{red} BDD}}
	\paragraph{}
		%Nous avons choisi d'utiliser phpmyadmin.
\newpage
\section{{\color{red} Client/Serveur}}

\newpage
\section{{\color{red} Scans}}

\newpage
\section{{\color{red} Interface WEB}}

\newpage
\section{{\color{red} Script }}
	\paragraph{Permissions}
		Bien que la majorité de nos scripts puissent s'exécuter avec les permissions d'un utilisateur, certain d'entre eux nécessites les droits d'administrateur.\\
	\subsection{{\color{blue}En Tant qu'utilisateur}}
		\begin{tabular}{|l|p{12cm}|}
			\hline
				\textbf{N°}&\textbf{Résultats}\\
			\hline
				1 & Hostname, Interfaces réseaux, nom de la distribution, version de la distribution, version du noyau, table de routage.\\
			\hline
				2 & Espace des partitions montées.\\
			\hline
				3 & Affiche les connections internet actives.\\
			\hline
				4 & Processus actif.\\
			\hline
				5 & Variables d'environnement.\\
			\hline
				6 & Informations CPU, Interruptions, Mémoire utilisée, Fichier\(s\) Swap\(s\), version du noyau,systèmes de fichiers montés, périphériques CPU, périphériques usb.\\
			\hline
				7 & Affiche les processus en cours dans une arborescence qui commence à la racine.\\
			\hline
				8 & Récupération de tous les fichiers d'extension ".log".\\
			\hline
				9 & Table de routage.\\
			\hline
				10 & interfaces réseaux.\\
			\hline
				11 & User loggé, heure du dernier démarrage, affiche les processus morts, runlevel courant.\\
			\hline
				13 & Liste des utilisateurs.\\
			\hline
				14 & Affiche l'état de la mémoire de la partition courante.\\
			\hline
				15 & Affichage de la derniére connexion local.\\
			\hline
				16 & Affiche les règles iptables pour filter, nat et mangle .\\
			\hline
				17 & Vérification de l'intérité de /bin, /usr/bin, /sbin, /usr/sbin.\\
			\hline
		\end{tabular}
			
	\subsection{{\color{blue}En Tant qu'administrateur}}
	
\newpage
\section{{\color{red}Conclusion}}

\newpage
\section{{\color{red}Évolution temporelle}}
\begin{tabular}{|l|p{12cm}|}
	\hline
		
		\textbf{DATE}& \textbf{OBJET} \\
		
	\hline
		10/09/14 & Présentation du projet et formation du binôme\\
	\hline
		17/09/14 & Début rapport. Étude du sujet. Installation VM Client\\
	\hline
		24/09/14 & Début client/server\\
	\hline
		01/10/14 & Rédaction des Makefiles de base\\
	\hline
		08/10/14 & Correction erreur client/serveur\\
	\hline
		07/11/14 & Fonction client exécution de script\\
	\hline
		10/11/14 & Premier scan. Création fonction de log client\\
	\hline
		19/11/14 & Autres script\\
	\hline
		03/12/14 & script mémoire + pattern script\\
	\hline
		08/12/14 & Scripts, ajout de la description des scripts au rapport\\
	\hline
		
		

\end{tabular}
		



\newpage
\section{{\color{red}Bibliographie}}
%\thispagestyle{empty}
	\subsection*{{\color{blue}C}}
		
	\subsection*{{\color{blue}SDL}}
		\paragraph{}
			
			
	\subsection*{{\color{blue}LaTeX}}
		\paragraph{}
	
%Une liste:
%
%\begin{itemize}
%\item un item
%\item un autre item
%\end{itemize}
%
%~\\
%
%Une énumération:
%\begin{enumerate}
%\item le premier item
%\item le deuxième
%\end{enumerate}
%
%
%\subsection{Installation du client}
%
%La figure~\ref{fig:tux} représente Tux, la mascotte de Linux.
%
%\begin{figure}[h!]
%	\centering
%	\includegraphics[scale=0.3]{images/tux.png}
%	\caption{Tux} 
%	\label{fig:tux}
%\end{figure}
%
%\subsection{Insérer du code}
%
%Le listing~\ref{list:helloWorld} donne un exemple de programme C.    
%    
%\begin{lstlisting}[caption=Hello World en C, label=list:helloWorld]
%#include <stdio.h>
%
%int main(int argc, char *argv[])
%{
%  printf("%s","Hello World!");
%  return 0;
%}
%\end{lstlisting}
%   
%%%%%%%%%%%%%%%%%%%%%%%%%%%%%%%%%%%%%%
%
%
%\newpage 
%\section{Divers}
%
%\subsection{Compilation}
%
%Installer Latex sur Ubuntu 13.04: %TODO installation sur la machine hôte
%\begin{lstlisting}
%$ sudo apt-get install texlive
%\end{lstlisting}
%
%Compiler le fichier .tex:
%\begin{lstlisting}
%$ pdflatex <FileName>.tex
%\end{lstlisting}
%
%Vous aurez besoin de compiler à deux reprises pour obtenir la table des matières et les références. 
%
%\subsection{Quelques liens utiles}
%
%\begin{itemize}
%\item \url{https://fr.wikibooks.org/wiki/LaTeX}
%\item \url{http://tobi.oetiker.ch/lshort/lshort.pdf}
%\end{itemize}


\end{document}
